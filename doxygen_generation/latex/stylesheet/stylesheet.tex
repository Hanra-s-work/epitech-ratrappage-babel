%% 
%% EPITECH PROJECT, $year
%% my_zappy 
%% File description:
%% stylesheet.tex
%%

\documentclass{article}
\usepackage[utf8]{inputenc} % For Unicode characters
\usepackage{graphicx} % For including images
\usepackage{hyperref} % For hyperlinks
\usepackage{fancyhdr} % For custom headers and footers
\usepackage{lipsum} % For generating dummy text (remove in actual document)

% Custom header and footer
\pagestyle{fancy}
\fancyhf{} % Clear header and footer
\rhead{Area - Doxygen - $year}
\lfoot{\thepage}
\rfoot{Area Project}

\begin{document}

% Title
\title{Area - Doxygen - 2024}
\author{Harleen Singh-Kaur, Thomas Lebouc, Eric Xu, Flavien Maillard, Henry Letellier}
\date{}
\maketitle

% Main content
\section{Introduction}
The Area project is a full stack consisting of a website and a mobile application.
The technologies used to create the project cannot be re-used in 2 different sections of the project.
This means that the website and the mobile app must be developed using 2 different technologies.
This also means that the server cannot use a technology used by the front (web, mobile).

The aim of Area is to Implement an IFTTT: (If this then that) action reaction type.
For example, if it is 9:00 send an e-mail to yourself reminding you to pack your lunchbox.

% Dummy text for demonstration purposes
\lipsum[1-10]

% Footer
\clearpage
\thispagestyle{empty} % No header or footer on this page
\begin{center}
    \small \textcopyright{} 2024 Area Game. All rights reserved. \\
    \small This is a project created in the Epitech education environment.
    \small They were the ones who asked us to create the project.
\end{center}

\end{document}
